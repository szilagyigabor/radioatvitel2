\chapter{Témalabor}

\section{Bevezetés}

A jelenleg zajló BSc képzésben, az önálló labort megelőző félévben Témalabort indítunk, mint ahogy az az előző félévi szakirány tájékoztatón is elhangzott.

Több más tanszékkel ellentétben (akik irodalomkutatást végeztetnek a hallgatókkal) mi a gyakorlatiasabb rész felé helyeznénk a hangsúlyt - többek között kedvcsinálónak illetve szintrehozónak szánva.

\section{A félév menete}

Ennek megfelelően a félév első felében - első 7 oktatási hét - mindegyik jelentkezett hallgató ugyanolyan áramkört épít meg dokumentáció és mérési utasítás alapján, az "Áramkörépítő szakkörös" pákákat illetve a "Lendületvételes" oszcilloszkópokat itt is hadrendbe állítva. \cite{hvthonlap}

A gyakorlati foglalkozások az első 7 oktatási héten, minden szerdán 8:15-12:00-ig lesznek megtartva a V1 épület 501/502 teremben. A megjelenés minden hallgatónak kötelező.

A félév második felében a hallgatók témakörök és laborok közül választhatnak, ahol egyéni vagy kisebb csoportokban a laborok által meghatározott feladatot kell elvégezniük.

Ezek a laborok és a hozzá tartozó oktatók, akiknél jelentkezni kell \aref{tab:laborok_es_oktatok} táblázatban láthatók.

\begin{table}[hb]
        \footnotesize
        \centering
        \caption{Oktatói táblázat}
        \begin{tabular}{ | l | l |}
        \hline
        Témakör & Oktató\\
        \hline
 		Antennák és EMC & dr. Nagy Lajos \\
 		DOCS & dr. Bitó János \\
 		EMT & dr. Pávó József \\
 		Mikrohullámú Távérzékelés & dr. Seller Rudolf \\
 		NES & Reichardt András \\
 		Űrtechnológia & dr. Csurgai-Horváth László \\
        \hline
        \end{tabular}
        \label{tab:laborok_es_oktatok}
\end{table}

\section{Követelmények}

\begin{enumerate}
\item
Az első 7 oktatási héten a gyakorlati foglalkozásokon a megjelenés kötelező, normál utcai viseletben.

Az áramkör előre gyártatott panelekre készül, kézi forrasztási eljárással (aki nem forrasztott még, az most megtanul), dokumentáció alapján.

Az egyes áramköri részegységek élesztése és bemérése (valamint mérés közbeni dokumentálása) az oktató kollégák segítségével történik digitális oszcilloszkópok felhasználásával.
\item
A félév második 7 hetes időszakában a hallgatók kisebb csoportokat alkotva egy-egy laborban tevékenykednek önállóan a laborok által meghatározott feladatokat elvégezve.

\item
A félév végén leadandó - a tanulmányi portálra feltöltendő - dokumentumok a következők:

\begin{itemize}
\item
A megadott sablon alapján készített mérési jegyzőkönyv a félév első felében épített áramkörről.
\item
5 - 10 oldal terjedelmű írásos beszámoló a félév második felében történt önálló munkáról csoportonként, a témát vezető oktató rövid értékelésével.
\item
A pótlási héten - felkészítve a hallgatókat a későbbi önálló labor beszámolókra, szakdolgozat illetve diplomaterv védésekre - 5 + 2 perces előadás jellegű beszámoló (LaTex pdf) a tárgyat hallgató hallgatók és az oktatók jelenlétében.
\end{itemize}

\end{enumerate}

A beszámolókhoz használható minta dokumentumok a következő linken érhetők el: \url{http://152.66.80.46/temalabor}

A félév végi érdemjegy a közös áramkör mérési jegyzőkönyve, a félév második felének írásos beszámolója és a félév pótlási hetében tartott 5 + 2 perces előadás alapján képződik egyenletes súlyozással 1-5-ig skálára kvantálva, ahol a \%-os határok a következők: 40, 55, 70, 85.

Például:
\begin{itemize}
\item
Mérési jegyzőkönyv: 85 \%
\item
Irásos beszámoló: 92 \%
\item
Előadás: 78 \%
\end{itemize}

Átlagban: (85+92+78)/3=85\% amely $\geq$ 85, vagyis jeles.


