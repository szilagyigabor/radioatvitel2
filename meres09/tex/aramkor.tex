\chapter{Grid-Dip Oscillator}

\section{Bevezetés}

A Grid-Dip Oscillator egy, a rádiózás hőskorából származó analóg mérőműszer, amely segítségével a következők paraméterek mérhetők (a teljesség igénye nélkül):

\begin{itemize}
\item
Egy adott passzív áramköri egység rezonancia frekvenciája: rezgőkör, szűrő, üregrezonátor, antenna, stb.
\item
Oszcillátorok vagy rádióadók frekvenciája.
\item
Kapacitás mérés (ismert induktivitással rezgőkörként).
\item
Induktivitás mérés (ismert kapacitással rezgőkörként).
\item
Rezonáns körök jósági tényezőjének összehasonlító mérése.
\item
Tápvonal rövidülési tényezője...
\end{itemize}

A GDO kapcsolási és nyomtatott áramköri rajza KiCAD tervezőprogramban készült, amely szabadon hozzáférhető és ingyenesen használható \cite{kicad}

\section{A GDO kapcsolási rajza}

Az elkészítendő áramkör kapcsolási rajza \aref{fig:gdoschematic} ábrán látható \cite{radiotechnika}.

\begin{figure}[!ht]
\centering
\includegraphics[width=150mm,keepaspectratio]{gdo_sch.pdf}
\caption{A GDO kapcsolási rajza}
\label{fig:gdoschematic} 
\end{figure}

Az áramkör alapvetően 3 jól elkülöníthető részegységből áll.

\begin{enumerate}
\item
A $Q_1$-gyel megvalósított Collpitts (kapacitív három-pont) kapcsolású nagyfrekvenciás L-C oszcillátor (rezgéskeltő), amely a $D_1$ és $D_2$ varikap diódák miatt az $RV_1$ potenciométerrel hangolható (feszültséggel változtatható a kimeneti periodikus jel frekvenciája).
\item
A $Q_2$, $Q_3$, $Q_4$ és $Q_6$ bipoláris tranzisztorokkal megvalósított munkaponti áramérzékelésre tervezett astabil multivibrátor, mint áramvezérelt hangfrekvenciás oszcillátor, amely a mért rezgő rendszerből kicsatolt RF jel, vagy a GDO, mint rezgő rendszerből kicsatolt energia hatására történő $Q_1$ munkapont eltolódását alakítja át emberi fül számára érzékelhető hangfrekvenciás jel változássá.
\item
A $Q_5$ segítségével megvalósított fázistolós R-C hangfrekvenciás oszcillátor, amely modulátorként működve a $Q_1$ Collpitts-oszcillátor frekvenciáját illetve amplitúdóját modulálja. Ennek célja, hogy a hagyományos szuperheterodin rendszerű rádió vevőkészülékek RF és KF fokozatainak mérése is lehetővé válhasson.
\end{enumerate}

\section{A GDO nyomtatott áramköri terve}

Az elkészítendő áramkör nyomtatott áramköri terve \aref{fig:gdopcb} ábrán látható.

A nyomtatott áramköri lapon jól felismerhető a Collpitts-oszcillátor tekercse, amely a varikap diódákkal alkot rezgőkört és határozza meg a rezgő rendszer rezonancia frekvenciáját.

Ezen tekercs menetszám megcsapolásokkal rendelkezik, melynek segítségével a GDO működési frekvencia tartománya így kb. 7 - 80 MHz-ig adódik. A változtatható tekercselés oka, hogy az alkalmazott kapacitás-diódák kapacitás átfogása 1:9, amely alapján a frekvencia átfogás (Thomson összefüggés) nem lehet nagyobb 1:3.

\begin{figure}[!ht]
\centering
\includegraphics[width=140mm,keepaspectratio]{gdo_pcb.pdf}
\caption{A GDO nyomtatott áramköri rajza}
\label{fig:gdopcb} 
\end{figure}

\section{A GDO alkatrész ültetési rajza}

Az elkészítendő áramkör alkatrész ültetési rajza \aref{fig:gdopos} ábrán látható.

Az ültetési rajzon nemcsak az alkatrész pozíciószámok, hanem az alkatrész értékek is könnydén beazonosíthatók.

A megvalósítandó áramkör egyaránt tartalmaz furatszerelt és felületszerelt áramköri elemeket. A felület-szerelt alkatrészeket a forrasztási oldalról, a furat-szerelt alkatrészeket pedig - az alkatrészlábak megfelelő mértékű meghajlítása után - az alkatrész ültetési oldalról kell a furatokon átdugva beültetni.

\begin{figure}[!ht]
\centering
\includegraphics[width=130mm,keepaspectratio]{gdo_pos.pdf}
\caption{A GDO alkatrész ültetési rajza}
\label{fig:gdopos} 
\end{figure}

\section{A GDO egyes részeinek szerelése és élesztése}

A GDO egyes részegységeinek beültetési és élesztési sorrendje a következő:

\begin{enumerate}
\item
Collpitts-oszcillátor.
\item
Munkaponti áramérzékelő hangfrekvenciás astabil multivibrátor a piezo hangsugárzóval.
\item
Modulátorként használt fázistolós R-C oszcillátor.
\end{enumerate}

A GDO egyes fokozatainak élesztése során mérni kell a következő paramétereket, amelyeknek a mérési jegyzőkönyvbe is bele kell kerülnie:

\begin{itemize}
\item
A GDO RF jelének frekvencia átfogása, a kimeneti RF jel szintje, jelalakja a legkisebb, legnagyobb frekvenciákon és az adott frekvencia tartomány közepén.
\item
A GDO áramérzékelőjének kimeneti jelének alakja, frekvenciája, amplitúdója az oszcillátor jelszint beállító potméterének ($RV_2$) két véghelyzetében és egy kb. középállásában 20 MHz kimeneti RF jel esetén.
\item
A GDO modulátor oszcillátorának kimeneti jele (amplitúdó, jelalak, frekvencia).
\end{itemize}

\section{Az elkészült GDO}

\subsection{Az alkatrész ültetési oldal}

Az elkészült GDO alkatrész ültetési oldalról készült fényképe látható \aref{fig:circtop} ábrán.

\begin{figure}[!ht]
\centering
\includegraphics[width=140mm,keepaspectratio]{circuit_top.pdf}
\caption{Az elkészült GDO az ültetési oldalról}
\label{fig:circtop} 
\end{figure}

\subsection{A forrasztási oldal}

Az elkészült GDO forrasztási oldalról készült fényképe látható \aref{fig:circbot} ábrán.

\begin{figure}[!ht]
\centering
\includegraphics[width=140mm,keepaspectratio]{circuit_bot.pdf}
\caption{Az elkészült GDO a forrasztási oldalról}
\label{fig:circbot} 
\end{figure}

