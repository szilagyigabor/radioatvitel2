\section{A mérési jegyzőkönyv formai követelményei}

\begin{itemize}
\item
A jegyzőkönyvben benne kell lennie, hogy mikor, milyen műszerrel történt a mérés (pl. a műszer gyári száma egyértelműen azonosítja a műszert) és ki vagy kik végezték a mérést.
\item
Minden logikailag összetartozó méréshez mérési elrendezési rajzot kell készíteni és rögzíteni kell az egyes beállított illetve mért paramétereket is. Ez azért szükséges, mert egy mérési jegyzőkönyv alapján reprodukálhatónak kell lennie a mérésnek, vagyis olyan részletességű kell legyen a mérési összeállítási rajz, hogy meg lehessen belőle állapítani, hogy melyik mérőponton, milyen beállítású mérőfejjel, hogyan történt a mérés.
\item
Minden mérési eredmény egyedileg vagy csoportosan röviden értékelni kell az adott részegységek paramétereinek megfelelően.
\item
A mérési összeállításról fényképfelvétel készítése is javasolt, amely helyettesítheti a mérési összeállítási rajzot, abban az esetben, ha minden eszköz illetve beállítás azon egyértelműen azonosítható.
\item
Ha valamilyen segédanyagot használtál (cikk, könyv, valakinek a diplomamunkája, Internetes segédanyag linkek, stb.), akkor azt az adott - logikailag megfelelő - helyen hivatkozd be, illetve tedd bele az irodalomjegyzék fájlba is. A Wikipédiás hivatkozások használatát mellőzd, ehelyett az ott hivatkozott irodalmat használd! - igaz ez mind a beszámolóra, mind a későbbi önlab doksikra, szakdolgozatokra és diplomatervekre egyaránt.
\end{itemize}

\subsection{A \LaTeX \ használatához további segítség}

Ha egy matematikai összefüggést kell leírnod (korrekt irodalmi hivatkozással \cite{diploma} ), akkor tedd például így: \ref{eq:antrow}.
% minden képlet esetén az align környzetet érdemes használni, mert ez a legújabb
\begin{align}
\ F ( \vartheta )= \sum_{k=0}^{N-1} I_k e^{-jk \beta d cos \vartheta} 
\label{eq:antrow}
\end{align}

Ahol:
\begin{itemize}
\item
$\vartheta$ a megfigyelési pont iránya az antennasorhoz képest
\item
$\beta$ a hullámszám (2$\pi$/$\lambda$)
\item
d az antennaelemek távolsága
\item
N az antennaelemek száma
\item
$I_k$ az aktuális antennaelem bemeneti komplex gerjesztése (árama)
\end{itemize}
Amennyiben kettő, vagy több egybe tartozó egyenletet kell leírni egyetlen környezetbe:
\begin{align}
\Delta \varphi &= 0 \\
\Delta \varphi &= -\frac{\varrho}{\varepsilon}
\end{align}
Az egyenlőség-jelhez igazított egyenleteket így lehet bevinni.
% Az '&' segítségével, oda rendezi az egymás felett levő egyenleteket, ahova az '&' jelet raktad.
% Figyelem! Az align környezet érzékeny az üres enterekre, ilyet ne hagyjunk, mert nem fordul a kód!
\\Ez itt egy függelék hivatkozás: \ref{fugg}

\subsection{Forráskódot}

például így társíthatsz LaTex-ben:

\lstinputlisting[language=C]{while1.c}